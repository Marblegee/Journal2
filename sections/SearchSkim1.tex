\section{Searching}
% \Url{}
\subsection{In-class Searching Exercise}
The following are the three papers reviewed in class:
\begin{enumerate}
    \item \url{https://academic.oup.com/bioinformatics/article/39/8/btad458/7232230}
    \item \url{https://www.sciencedirect.com/science/article/pii/S2542660523000148}
    \item \url{https://www.jair.org/index.php/jair/article/view/14649}
\end{enumerate}

In this phase, I immediately eliminated the paper on patient monitoring system because it has no correlation whatsoever with my area of research. Furthermore, I had the DP-fy paper discarded due to the incongruent nature of the abstract, despite suggesting some machine learning inputs. Lastly, the ScHiCEDRN paper was my choice for skimming since it suggests advancement in bioinformatics and deep learning at a glance and is very much related to my research domain.

\subsection{Extensive Searching and Skim Exercise}
Sequel to my domain-centered academic paper search exercise aimed at fine-tunning my familiarity with the Google Scholar platform, the following are descriptions of five papers chosen for my skimming exercise in no particular order.

DeepHiC \cite{hong_deephic_2020} is a deep learning-based methodology that boasts of a 22-layer neural network, efficient enough to recover a 10-kilobase HiC data from a $1\%$ DNA sequencing ratio, surpassing previous models in performance. Furthermore, it introduced a novelty in the HiC resolution enhancement domain through its alloy of multiple loss functions for training.

hicGAN\cite{liu_hicgan_2019} is one of the earlier models in the HiC fidelity restoration domain, a pioneer model whose architecture creatively leverages Generative Adversarial Networks (GAN) to foster a refinement in the restoration of HiC resolutions and easing the identification of genomic topologically associating domains.

In an attempt to redress the sequencing cost required to obtain a high-quality HiC data, GraphHiC \cite{murtaza_graphic_2023} introduces a novel approach of adopting a graph representation for HiC data combined with the more readily available ChiP-seq. This is definitely a worthy read as it boasts of dual novelty.

It is no surprise that HiCARN \cite{hicks_hicarn_2022} is featured in "Bioinformatics," the most revered journal in the field of Computational Biology. HiCARN formulated two computationally valid approaches to the HiC resolution problem leveraging GAN and cascading residual CNN respectively. 

The VEHiCLE \cite{highsmith_vehicle_2021} is a deep neural network framework that fuses a variational autoencoder and adversarial training strategy to solve the resolution enhancement problem of HiC contact maps in the genomic structure research domain. This material is particularly a worthy read because of its unusual dose of novelty.